% !TEX TS-program = pdflatex
% !TEX encoding = UTF-8 Unicode

% This is a simple template for a LaTeX document using the "article" class.
% See "book", "report", "letter" for other types of document.

\documentclass[10pt]{article} % use larger type; default would be 10pt

\usepackage[utf8]{inputenc} % set input encoding (not needed with XeLaTeX)


%%% Examples of Article customizations
% These packages are optional, depending whether you want the features they provide.
% See the LaTeX Companion or other references for full information.

%%% PAGE DIMENSIONS
\usepackage{geometry} % to change the page dimensions
\geometry{a4paper} % or letterpaper (US) or a5paper or....

\usepackage{graphicx} % support the \includegraphics command and options
\usepackage{amsmath,amsthm,amssymb}
\DeclareMathOperator*{\argmax}{arg\,max}

%%% PACKAGES
\usepackage{booktabs} % for much better looking tables
\usepackage{array} % for better arrays (eg matrices) in maths
\usepackage{paralist} % very flexible & customisable lists (eg. enumerate/itemize, etc.)
\usepackage{verbatim} % adds environment for commenting out blocks of text & for better verbatim
\usepackage{subfig} % make it possible to include more than one captioned figure/table in a single float
% These packages are all incorporated in the memoir class to one degree or another...

%%% HEADERS & FOOTERS
\usepackage{fancyhdr} % This should be set AFTER setting up the page geometry
\pagestyle{fancy} % options: empty , plain , fancy
\renewcommand{\headrulewidth}{0pt} % customize the layout...
\lhead{}\chead{}\rhead{}
\lfoot{}\cfoot{\thepage}\rfoot{}

%%% SECTION TITLE APPEARANCE
\usepackage{sectsty}
\allsectionsfont{\sffamily\mdseries\upshape} % (See the fntguide.pdf for font help)
% (This matches ConTeXt defaults)

%%% ToC (table of contents) APPEARANCE

\usepackage{graphicx}
\usepackage{pstricks-add}
\usepackage{auto-pst-pdf}
\usepackage{pst-pdf}

\title{Triplet Network - First Report}
\author{Elad Hoffer}
%\date{} % Activate to display a given date or no date (if empty),
         % otherwise the current date is printed

\begin{document}
\maketitle
\section{Triplet Network}
\subsection{Overview}
\emph{Triplet network} (inspired by "Siamese network") is comprised of 3 instances of the same feed-forward network (with shared parameters). \\
When fed with 3 samples, the network
outputs 2 values - the $L_2$ distance between the embedded representation of 2 input from the representation of the third. \\
If we will denote the 3 inputs as $x$ $y_1$ and $y_2$, and the
embedded representation of the network as $Net(x)$, the output will be the vector 
\begin{equation*}
    TripletNet(x,y_1,y_2)= \begin{bmatrix}
                       \|Net(x)-Net(y_1)\|_2 \\[0.3em]
                      \|Net(x)-Net(y_2)\|_2 \\

                      \end{bmatrix}
\end{equation*}
\subsection{Training}
Training is preformed by feeding the network with random samples, where $x$ and $y^{+}$ are of the same class, and $y^{-}$ is of different class.
The objective is to classify correctly which sample is of the same class as $x$. Two same-class samples should have a lower $L_2$ distance after the network embedding. In order to distinguish between the highest distance, a SoftMax function is applied on both outputs - effectively creating a ratio measure.\\
By using the same shared-parameters network, we allow the back-propagation algorithm to update the model with regard to all samples \emph{simultaneously}. Training is done
by simple stochastic-gradient-descent on a negative-log-likelihood loss with regard to the 2-class problem. 


\begin{figure}
\begin{center}
\psscalebox{1.0 1.0} % Change this value to rescale the drawing.
{
\begin{pspicture}(0,-5.2)(8.5,5.2)
\psframe[linecolor=black, linewidth=0.04, dimen=outer](2.65,-3.6)(1.05,-5.2)
\psframe[linecolor=black, linewidth=0.04, dimen=outer](5.05,-3.6)(3.45,-5.2)
\psframe[linecolor=black, linewidth=0.04, dimen=outer](7.55,-3.6)(5.95,-5.2)
\psframe[linecolor=black, linewidth=0.04, dimen=outer](3.8,3.2)(0.0,2.0000067)
\psframe[linecolor=black, linewidth=0.04, dimen=outer](5.35,5.2)(3.15,4.4)
\rput[bl](0.25,2.4){$\|Net(x)-Net(x^{-})\|_2$}
\psframe[linecolor=black, linewidth=0.04, dimen=outer](8.5,3.2)(4.7,2.0000067)
\rput[bl](4.95,2.4){$\|Net(x)-Net(x^{+})\|_2$}
\psframe[linecolor=black, linewidth=0.04, dimen=outer](2.65,0.6)(1.05,-2.8)
\psframe[linecolor=black, linewidth=0.04, dimen=outer](5.05,0.6)(3.45,-2.8)
\psframe[linecolor=black, linewidth=0.04, dimen=outer](7.45,0.6)(5.85,-2.8)
\psline[linecolor=black, linewidth=0.04](1.85,0.6)(1.85,2.0)(4.25,0.6)(6.75,2.0)(6.75,0.6)(6.75,0.6)
\psline[linecolor=black, linewidth=0.04](1.85,-3.6)(1.85,-2.8)(1.85,-2.8)
\psline[linecolor=black, linewidth=0.04](4.25,-3.6)(4.25,-2.8)(4.25,-2.8)
\psline[linecolor=black, linewidth=0.04](6.75,-3.6)(6.75,-2.8)(6.75,-2.8)
\rput[bl](1.75,-4.5){$x^{-}$}
\rput[bl](4.15,-4.5){$x$}
\rput[bl](6.65,-4.5){$x^{+}$}
\psline[linecolor=black, linewidth=0.04](1.85,3.2)(4.25,4.4)(6.65,3.2)(6.65,3.2)
\rput[bl](3.3,4.6){$Comparator$}
\rput[bl](1.45,-1.1){$Net$}
\rput[bl](3.95,-1.1){$Net$}
\rput[bl](6.35,-1.1){$Net$}

\end{pspicture}
}
 \caption{Triplet Network}\label{TripletNet}
 \end{center}
\end{figure}

\
\end{document}
